\chapter{ΕΡΩΤΗΜΑ 3}

    Τα γενικευμένα δέντρα επιθεμάτων (generalized suffix trees) επιτρέπουν την αποθήκευση και την αναζήτηση πολλαπλών συμβολοσειρών, εν αντιθέσει με τα δέντρα επιθεμάτων (suffix trees).
    Πρόκειται για μια \textbf{στατική δομή δεδομένων}, μιας και κατασκευάζεται για κάποιες συγκεκριμένες συμβολοσειρές που ορίζονται εξ' αρχής.
    Ως αποτέλεσμα, η δομή δεν έχει σχεδιαστεί για να δέχεται εύκολα τροποποιήσεις, όπως είναι η εισαγωγή νέων συμβολοσειρών ή η διαγραφή υπάρχοντων. Συγκεκριμένα:

    \section{ΠΡΟΒΛΗΜΑΤΑ ΣΤΑΤΙΚΩΝ ΔΕΝΤΡΩΝ}
        \subsection{Πολυπλοκότητα στην τροποποίηση εισόδου}
            Για να μπορέσει να εισαχθεί μια νέα συμβολοσειρά στο γενικευμένο δέντρο επιθεμάτων, είναι απαραίτητη η ανακατασκευή ολόκληρου του δέντρου μιας και άλλαξε η είσοδος.
            Έτσι όλα τα υπάρχοντα μονοπάτια --που αναπαριστούν τα υπάρχοντα επιθέματα-- χρειάζεται να ανανεωθούν για να συμβαδίζουν με τις αλλαγές της εισόδου.

            Παρόμοια ανανέωση απαιτείται και με διαγραφή κάπιας συμβολοσειράς, αφού τροποποιείται και πάλι η είσοδος του δέντρου.
%            Μάλιστα η διαγραφή μιας υπάρχουσας συμβολοσειράς είναι ακόμα πιο επικίνδυνη, μιας και μπορεί να ακυρώσει υπάρχοντα μονοπάτια, επομένως απαιτείται μια μεγάλη αναδιάρθρωση ώστε όλα τα επιθέματα να αναπαρίστανται σωστά.

        \subsection{Χρονική πολυπλοκότητα}
            Η αναδιάρθρωση των μονοπατιών από την αρχή μετά από κάθε εισαγωγή και διαγραφή δεν είναι αποδοτική καθώς κοστίζει \(O(n)\) χρόνο ανά τροποποίηση (\(n\): συνολικός αριθμός χαρακτήρων των συμβολοσειρών)

    \section{ΔΥΝΑΜΙΚΟΙ ΑΛΓΟΡΙΘΜΟΙ}
        Είναι σαφές ότι είναι απαραίτητος ένας δυναμικός τρόπος διαχείρισης της δομής, ώστε να μη χρειάζεται η ανακατασκευή όλων των μονοπατιών κάθε φορά που αλλάζει η είσοδος του δέντρου, αλλά παρά μόνο των μονοπατιών που επηρεάζονται.

        \subsection{Δυναμικό δέντρο επιθεμάτων του McCreight}
            Ο McCreight προτείνει έναν νέο αλγόριθμο \cite{McCreight_1976}, ο οποίος κατασκευάζει το δέντρο επιθεμάτων σταδιακά (Algorithm M), προσθέτοντας ένα επίθεμα τη φορά, από το μεγαλύτερο στο μικρότερο.
            Κατ' αυτόν τον τρόπο, αν η είσοδος αλλάξει, πλέον είναι δυνατό να ανανεωθεί μόνο το κομμάτι του δέντρου που επηρεάζεται από την αλλαγή.
            Με τη καινοτομία του McCreight, τα suffix links (ένας πρόσθετος δείκτης που βοηθάει στην εύρεση της θέσης που θα εισαχθεί το επόμενο επίθεμα) έχουν δημιουργηθεί, βοηθούν τον αλγόριθμο να βρει εύκολα πού χρειάζεται να γίνουν αλλαγές, και έτσι δεν είναι απαραίτητη η ανακατασκευή του δέντρου από την αρχή.

        \subsection{Δυναμικό δέντρο επιθεμάτων των Choi - Lam}
            Οι Choi - Lam προτείνουν μια νέα υλοποίηση για το δέντρο επιθεμάτων, με την ίδια πολυπλοκότητα με τα στατικά δέντρα, αλλά με τη διαφορά ότι το δέντρο πλέον δεν διατηρεί αποθηκευμένες συμβολοσειρές που διαγράφονται. \cite{Choi_Lam_1997}

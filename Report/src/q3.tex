\chapter{ΕΡΩΤΗΜΑ 3}

    Τα γενικευμένα δέντρα επιθεμάτων (generalized suffix trees) επιτρέπουν την αποθήκευση και την αναζήτηση πολλαπλών συμβολοσειρών, εν αντιθέσει με τα δέντρα επιθεμάτων (suffix trees) που αφορούν μια συγκεκριμένη συμβολοσειρά.
    Πρόκειται για μια \textbf{στατική δομή δεδομένων}, μιας και κατασκευάζεται για κάποιες συγκεκριμένες συμβολοσειρές που ορίζονται εξ' αρχής.
    Ως αποτέλεσμα, η δομή δεν έχει σχεδιαστεί για να δέχεται εύκολα τροποποιήσεις, όπως είναι η εισαγωγή νέων συμβολοσειρών ή η διαγραφή υπάρχοντων. Συγκεκριμένα:

    \section{ΠΡΟΒΛΗΜΑΤΑ ΣΤΑΤΙΚΩΝ ΔΕΝΤΡΩΝ}
        Για να μπορέσει να εισαχθεί μια νέα συμβολοσειρά στο γενικευμένο δέντρο επιθεμάτων, είναι απαραίτητη η ανακατασκευή ολόκληρου του δέντρου μιας και άλλαξε η είσοδος.

        Έτσι όλα τα υπάρχοντα μονοπάτια --που αναπαριστούν τα υπάρχοντα επιθέματα-- χρειάζεται να ανανεωθούν για να συμβαδίζουν με τις αλλαγές της εισόδου.
        Παρόμοια ανανέωση απαιτείται και με διαγραφή κάποιας συμβολοσειράς, αφού τροποποιείται και πάλι η είσοδος του δέντρου.

        Η αναδιάρθρωση των μονοπατιών από την αρχή μετά από κάθε εισαγωγή και διαγραφή δεν είναι αποδοτική καθώς κάθε φορά είναι απαραίτητο να επαναυπολογιστεί ολόκληρο το δέντρο.

    \section{ΔΥΝΑΜΙΚΟΙ ΑΛΓΟΡΙΘΜΟΙ}
        Είναι σαφές ότι είναι απαραίτητος ένας δυναμικός τρόπος διαχείρισης της δομής, ώστε να μη χρειάζεται η ανακατασκευή όλων των μονοπατιών κάθε φορά που αλλάζει η είσοδος του δέντρου, αλλά παρά μόνο των μονοπατιών που επηρεάζονται.

        \subsection{Δυναμικό δέντρο επιθεμάτων του McCreight}
            Ο McCreight προτείνει έναν νέο αλγόριθμο \cite{McCreight_1976}, ο οποίος κατασκευάζει το δέντρο επιθεμάτων σταδιακά (Algorithm M), προσθέτοντας ένα επίθεμα τη φορά.
            Κατ' αυτόν τον τρόπο, δεν είναι απαραίτητη η πρότερη γνώση όλων των συμβολοσειρών, συντελώντας σε μια \textit{κάπως} δυναμική μορφή δέντρου, από την άποψη ότι δε χρειάζεται η πρωτύτερη γνώση ολόκληρης της εισόδου για να ξεκινήσει η εισαγωγή των επιθεμάτων.

            Προφανώς, η πρόταση του McCreight δεν αποτελεί μια καθαρόαιμη δυναμική δομή δεδομένων.

        \subsection{Δυναμικό δέντρο επιθεμάτων των Choi - Lam}
            Οι Choi - Lam προτείνουν μια νέα δυναμική υλοποίηση για το δέντρο επιθεμάτων. \cite{Choi_Lam_1997}
            Κατά την εισαγωγή μιας νέας συμβολοσειράς, το δέντρο ψάχνει το μεγαλύτερο επίθεμα και μετά προσθέτει τα νέα επιθέματα κάνοντας τις απαραίτητες μεταβολές στις ακμές και στους κόμβους.
            Στην εισαγωγή χρησιμοποιείται μια ιδέα που παρουσίασε και ο McCreight, τα suffix links.

            Αντίστοιχα κατά τη διαγραφή μιας συμβολοσειράς \(s\), αναγνωρίζονται και διαγράφονται οι ακμές / επιθέματα που σχετίζονται με το \(s\) και ο αλγόριθμος ανανεώνει τις ετικέτες των φύλλων τους.
            Το αποτέλεσμα είναι σε κάθε περίπτωση να επανακατασκευάζεται το δέντρο ακέραια, επιτρέποντας τη δυναμική προσθήκη και διαγραφή συμβολοσειρών σε \(O(nlogA)\) χρόνο.

        \subsection{Online αλγόριθμος του Ukkonen}
            O Ukkonen προτείνει έναν νέο αλγόριθμο κατασκευής δέντρων επιθεμάτων σε γραμμικό χρόνο. \cite{Ukkonen_1995}
            Ο αλγόριθμος διαχειρίζεται την εισαγωγή και διαγραφή των συμβολοσειρών με έναν τρόπο που επιτρέπει την ανανέωση του δέντρου χωρίς να χρειάζεται η ανακατασκευή όλων των μονοπατιών.

            Για κάθε νέο χαρακτήρα που εισάγεται, ο αλγόριθμος ανανεώνει το δέντρο επιθεμάτων επεκτείνοντας τα υπάρχοντα επιθέματα με τον νέο χαρακτήρα.
            Όταν διαγράφεται μια συμβολοσειρά, ο αλγόριθμος διασχίζει το δέντρο για να εντοπίσει και να διαγράψει τους κόμβους που αντιστοιχούν στα επιθέματα του διαγραμμένου χαρακτήρα.
            Αυτά επιτυγχάνονται σε γραμμικό χρόνο.

        \subsection{LCP αλγόριθμος των Cole - Hariharan}
            Τέλος οι Cole - Hariharan προτείνουν έναν LCP (Longest Common Prefix) αλγόριθμο για το δέντρο επιθεμάτων, με \(O(\log n)\) χειρότερο χρόνο για εισαγωγές και διεγραφές. \cite{Cole_Hariharan_2005}

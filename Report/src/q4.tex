\chapter{ΕΡΩΤΗΜΑ 4}
    \vspace{-10pt}
    \begin{center} \noindent
        \includegraphics[scale=0.4]{img/erwthma4_example}
    \end{center}

    Ένα \textbf{Γενικευμένο Δέντρο Επιθεμάτων} περιέχει όλα τα suffixes από πολλαπλές συμβολοσειρές.
    Κάθε κόμβος του δέντρου αναπαριστά ένα κοινό suffix από αυτές τις συμβολοσειρές, και τα φύλλα πληροφορίες για τις συμβολοσειρές και τη θέση του επιθέματος μέσα στην συμβολοσειρά.

    Επομένως, για την εύρεση επαναλήψεων, διασχίζουμε το δέντρο και συλλέγουμε τους κόμβους/φύλλα.
    Αν ένας κόμβος περιλαμβάνει \(k\) φορές το ίδιο επίθεμα, τότε αυτή η συμβολοσειρά επαναλαμβάνεται σε διαφορετικές ακολουθίες.
    Σε περίπτωση που υπάρχουν περιορισμοί στα κενά, τότε θα λάβουμε υπόψη μας και τις θέσεις των επιθεμάτων, όπως είναι αποθηκευμένες στα φύλλα.


    Με χρήση του \textbf{K-mers} αλγορίθμο μπορούμε να σπάσουμε τη συμβολοσειρά σε μικρότερα κομμάτια μήκους \(k\), όπου \(k\) το μέγεθος της συμβολοσειράς που επαναλαμβάνεται, και να συγκρίνουμε τα σπασμένα κομμάτια της με την επαναλαμβανόμενη.
    Για πιο αποδοτική σύγκριση μπορούμε να χρησιμοποιήσουμε κάποιο hashtable.
    Στην περίπτωση που υπάρχουν περιορισμοί στα κενά μπορούμε να προσπεράσουμε κάποιες συγκρίσεις συμβολοσειρών όταν έχουμε βρει match.


%
%    \section{ΜΕ ΠΕΡΙΟΡΙΣΜΟΥΣ ΣΤΑ ΚΕΝΑ}
%    Στην περίπτωση που υπάρχουν ειδικές συνθήκες στο μεσοδιάστημα ανάμεσα στις συμβολοσειρές που επαναλαμβάνονται (για παράδειγμα αν υπάρχει κάποιος περιορισμός των χαρακτήρων), ο αλγόριθμος πρέπει να διασφαλίσει τα τους περιορισμούς στα κενά:
%
%    Πάλι διασχίζουμε το γενικευμένο binarized δέντρο επιθεμάτων από κάτω προς τα πάνω και περνάμε από ζευγάρια ακμών.
%    Η διαφορά είναι πως πλέον ταξινομούμε τα leaf lists.
%    Η ταξινόμηση είναι σημαντική, μιας και έτσι διασφαλίζεται η αποδοτική εύρεση δυνητικών συμβολοσειρών που ικανοποιούν τους περιορισμούς στα κενά που θέσαμε.
%    Η υπόλοιπη διαδικασία είναι η ίδια με την προηγούμενη.


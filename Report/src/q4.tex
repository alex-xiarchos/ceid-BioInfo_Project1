\chapter{ΕΡΩΤΗΜΑ 4}
    Σκοπός είναι η εύρεση των μέγιστων multirepeats (συμβολοσειρά που εμφανίζεται πολλαπλές σειρές) σε μια συλλογή ακολουθιών: \cite{Bakalis_2006}
    \vspace{-10pt}
    \begin{center} \noindent
        \includegraphics[scale=0.4]{img/erwthma4_example}
    \end{center}

    \section{ΧΩΡΙΣ ΠΕΡΙΟΡΙΣΜΟΥΣ ΣΤΑ ΚΕΝΑ}
    Θα χρησιμοποιήσουμε γενικευμένο δέντρο επιθεμάτων στο οποίο κάθε κόμβος έχει δύο το πολύ παιδιά (binarizated).
    Διασχίζοντας το δέντρο από τα φύλλα προς τη ρίζα, περνάμε ταυτόχρονα από κόμβους με κοινό πατέρα.

    Έτσι για κάθε ζευγάρι των κόμβων \(v\) και \(w\) αναζητούνται multirepeats που δημιουργούνται από τα κοινά προθέματα των επιθεμάτων των κόμβων, και αν βρεθούν, συνενώνονται σε leaf lists.
    Τα leaf lists περιλαμβάνουν όλες τις εμφανίσεις των επιθεμάτων στο υποδέντρο που εξετάζουμε.
    Αυτές, συνενώνονται και αντιστοιχίζονται στον πατέρα κόμβο \(z\).
    Η διαδικασία επαναλαμβάνεται προς τα πάνω.

    \section{ΜΕ ΠΕΡΙΟΡΙΣΜΟΥΣ ΣΤΑ ΚΕΝΑ}
    Στην περίπτωση που υπάρχουν ειδικές συνθήκες στο μεσοδιάστημα ανάμεσα στις συμβολοσειρές που επαναλαμβάνονται (για παράδειγμα αν υπάρχει κάποιος περιορισμός των χαρακτήρων), ο αλγόριθμος πρέπει να διασφαλίσει τα τους περιορισμούς στα κενά:

    Πάλι διασχίζουμε το γενικευμένο binarized δέντρο επιθεμάτων από κάτω προς τα πάνω και περνάμε από ζευγάρια ακμών.
    Η διαφορά είναι πως πλέον ταξινομούμε τα leaf lists.
    Η ταξινόμηση είναι σημαντική, μιας και έτσι διασφαλίζεται η αποδοτική εύρεση δυνητικών συμβολοσειρών που ικανοποιούν τους περιορισμούς στα κενά που θέσαμε.
    Η υπόλοιπη διαδικασία είναι η ίδια με την προηγούμενη.
\chapter{ΕΡΩΤΗΜΑ 1}

    \section{ΕΡΓΑΛΕΙΑ ΓΙΑ ΧΕΙΡΙΣΜΟ ΠΡΟΒΛΗΜΑΤΩΝ ΒΙΟΠΛΗΡΟΦΟΡΙΚΗΣ}

        Η σελίδα της Rosalind περιλαμβάνει κάποια βασικά προβλήματα, με σκοπό μια πρώτη εξοικείωση στο τομέα της Βιοπληροφορικής.

    \subsection{"Introduction to the Bioinformatics Armory": (SMS 2)}
        Το πρώτο πρόβλημα αφορά την εύρεση των αριθμών των νουκλεοτιδίων από μια ακολουθία DNA.
        Ένα εργαλείο για την ανάλυση της ακολουθίας είναι το Sequence Manipulation Suite (SMS) 2.
        Πρόκειται για μια συλλογή Javascript προγραμμάτων για τη δημιουργία, στοίχιση και ανάλυση μικρών DNA και πρωτεϊνικών ακολουθιών. \cite{SMS2}
        Χρησιμοποιώντας το DNA Stats, εισάγουμε το Sample Dataset και εισάγεται το πλήθος των νουκλεοτιδίων, και το ποσοστό εμφάνισής τους:
        \vspace{-10pt}
        \begin{table}[ht] \noindent\centering\tt
        \resizebox{0.4\textwidth}{!}{
            \begin{tabular}{lll}
            Pattern & Times found: & Percentage \\
            \midrule
            g & 17 & 24.29 \\
            a & 20 & 28.57 \\
            c & 21 & 30 \\
            3 & 12 & 17.14 \\
            \end{tabular}}
        \end{table}
        \vspace{-10pt}

    \subsection{"GenBank Introduction": Αναζήτηση}
        Αφορά τη βάση δεδομένων GenBank. \cite{GenBank} Μπορούμε να αναζητήσουμε ακολουθίες νουκλεοτιδίων και πρωτεϊνών, όπως επίσης και βιβλιογραφικές δημοσιεύσεις.

    \subsection{"Data Formats": Formats της GenBank}
        Στην GenBank για να υπάρχει μια συνέπεια στην αναπαράσταση των νουκλεοτιδικών ακολουθιών ακολουθείται ένα συγκεκριμένο format που περιλαμβάνει το header, τα χαρακτηριστικά της ακολουθίας και την ίδια την ακολουθία.
        Το εργαλείο GenBank to Fasta του SMS 2 \cite{GenBankToFASTA} επιτρέπει την αντιγραφή κάποιου entry από το GenBank και τη μετατροπή του σε FASTA, τη πιο δημοφιλέστερη μορφή αναπαράστασης ακολουθιών.
        \footnote{Η μορφή FASTA αποτελείται από ένα header (που ξεκινάει με το σύμβολο ">" και ένα αγνωριστικό της ακολουθίας), και τα δεδομένα της αλληλουχίας.
                Κάθε εγγραφή της GenBank είναι πολύ πιο λεπτομερής σε σχέση με τη FASTA: περιλαμβάνει πληροφορίες όπως το μήκος της ακολουθίας, ημερομηνία τροποποίησης, περιγραφή της ακολουθίας, χαρακτηριστικά της ακολουθίας όπως επίσης και δημοσιεύσεις που σχετίζονται με την ακολουθία.}
        Για παράδειγμα:

\begin{graycomment} \footnotesize
    \begin{verbatim}
GenBank to FASTA results
>Strongylocentrotus purpuratus fascin (FSCN1) mRNA, complete cds.
acttgaaagtggataaaatcgactgataccaaaacaacattgttttacagaagtggtcgt
ttgaggacatcaacatatttcacaatgcctgctatgaatttaaaatacaaatttggcctg\end{verbatim}
\end{graycomment}

    \subsection{"New Motif Discovery": Αναζήτηση Motifs σε ακολουθίες}
        Με το εργαλείο MEME (Multiple Em for Motif Elicitation) \cite{MEME}, εισάγοντας ακολουθίες που περιλαμβάνει motif
        \footnote{Πρόκειται για ένα μοτίβο συγκεκριμένης λειτουργικής σημασίας που επαναλαμβάνεται σε μια ακολουθία.
                Ο εντοπισμός τους έχει σημασία για την κατανόηση της λειτουργίας των αλληλουχιών.}
        , εξάγεται η κανονική έκφραση (regular expression) του συγκεκριμένου motif.

            \begin{graycomment} \footnotesize
            \begin{verbatim}
Motif TISWYQ MEME-1 regular expression
----------------------------------------------------------
TISWYQ

Motif YQPARIKEFAK MEME-2 regular expression
----------------------------------------------------------
[YQ][ALQ][PC][AGV]R[IR][KV][ERS]F[AMN][KC] \end{verbatim}
            \end{graycomment}

    Η κανονική έκφραση του μοτίβου που εξήχθη από τη δεύτερη ακολουθία για παράδειγμα, έχει ως πρώτη θέση πάντα το Y ή το Q, στη δεύτερη θέση πάντα τη A, L ή Q,

    \subsection{"Pairwise Global Alignment": Στοίχιση ακολουθιών}
        Στο εργαλείο Needle \cite{Needle} μπορούμε να εισάγουμε τα ID από δύο GenBank entries, με σκοπό την ολική στοίχισή τους.

        Κομμάτι του αποτελέσματος που εξάγεται:
            \begin{graycomment} \footnotesize
            \begin{verbatim}
%# Length: 142
%# Identity:     122/142 (85.9%)
%# Similarity:   131/142 (92.3%)
%# Gaps:           0/142 ( 0.0%)
%# Score: 648.0 \end{verbatim}
            \end{graycomment}

    \subsection{"FASTQ format introduction": Μετατροπή FASTQ σε FASTA}
        Η διαφορά του FASTQ αρχείου σε σχέση με το FASTA είναι πως περιλαμβάνει quality scores (πληροφορίες ποιότητας) για κάθε νουκλεοτίδιο στην ακολουθία.
        Αναπαρίσταται ως μια "γραμμή ποιότητας", κάτω από την ακολουθία, συνδεδμένη με ένα "+".

        Υπάρχουν διαφορετικοί online convertors που μπορούν να το μετατρέψουν σε FASTA, όπως ο Sequence Conversion της Bugaco, \cite{BugacoConversion} στον οποίον ανεβάζουμε ένα FASTQ αρχείο και το μετατρέπουμε σε αρχείο \texttt{.fasta}.

    \subsection{"Read Quality Distribution": Per sequence quality analysis}
        Το FastQC \cite{FastQC} είναι λογισμικό ανάγνωσης ακολουθιακών δεδομένων, το οποίο μπορεί να εξάγει γραφικά και πίνακες ελέγχου ποιότητας των ακολουθιών.
    \begin{graycomment} \footnotesize
    \begin{verbatim}
INPUT:
    @Rosalind_0041
    GGCCGGTCTATTTACGTTCTCACCCGACGTGACGTACGGTCC
    +
    6.3536354;.151<211/0?::6/-2051)-*"40/.,+%)
    @Rosalind_0041
    TCGTATGCGTAGCACTTGGTACAGGAAGTGAACATCCAGGAT
    +
    AH@FGGGJ<GB<<9:GD=D@GG9=?A@DC=;:?>839/4856
    @Rosalind_0041
    ATTCGGTAATTGGCGTGAATCTGTTCTGACTGATAGAGACAA
    +
    @DJEJEA?JHJ@8?F?IA3=;8@C95=;=?;>D/:;74792\end{verbatim}
    \end{graycomment}
    \begin{figure}[H] \noindent \centering
        \includegraphics[scale=0.55]{img/FastQC}
        \caption{Γραφικό περιβάλλον FastQC.}
    \end{figure}

    \subsection{"Protein Translation": SMS 2 Translate}
        Μέσω του εργαλείου Translate του SMS 2 \cite{Translate}, μπορούμε να μεταφράσουμε την αλληλουχία των νουκλεοτιδίων σε αμινοξέα. Για παράδειγμα:

\begin{graycomment} \footnotesize
    \begin{verbatim}
INPUT:
    >test
    ATGGCCATGGCGCCCAGAACTGAGATCAATAGTACCCGTATTAACGGGTGA
OUTPUT:
    >rf 1 test
    MAMAPRTEINSTRING*
\end{verbatim}
\end{graycomment}

    \subsection{"Read Filtration by Quality": FASTQ Quality Filter}
        Μπορούμε να "καθαρίσουμε" αδιάφορα κομμάτια από τις αλληλουχίες, με στόχο τη βελτίωση της ποιότητας των δεδομένων μας, χρησιμοποιώντας το FASTQ Quality Filter της Galaxy. \cite{GalaxyFastQQualityFliter}

        Από το εργαλείο εξάγεται το αρχείο \verb|Galaxy2-[Filter_by_quality_on_data_1].fastqsanger| το οποίο περιλαμβάνει μόνο τα φιλταρισμένα entries.

    \subsection{"Complementing a Strand of DNA": SMS 2 Reverse Complement }
        Το Reverse Complement του SMS 2 επιστρέφει τα συμπληρωματικά νουκλεοτίδια. Για παράδειγμα:
\begin{graycomment} \footnotesize
    \begin{verbatim}
INPUT:
    >Rosalind_12
    GACTCCTTTGTTTGCCTTAAATAGATACATATTTACTCTTGACTCTTTT...
    ...GTTGGCCTTAAATAGATACATATTTGTGCGACTCCACGAGTGATTCGTA
    >Rosalind_37
    ATGGACTCCTTTGTTTGCCTTAAATAGATACATATTCAACAAGTGTGCA...
    ...CTTAGCCTTGCCGACTCCTTTGTTTGCCTTAAATAGATACATATTTG
OUTPUT:
    The best non-identical alignments are:     ls-w bits E(1) %_id  %_sim  alen
    Rosalind_37                     (  96) [f]  465 35.8 1.6e-07 0.763 0.774   93
    +-                                          308 19.1   0.017 0.549 0.593   91
    +-                                          252 13.1    0.65 0.476 0.563  103
    +-                                          244 12.3    0.85 0.489 0.564   94
    +-                                          235 11.3    0.98 1.000 1.000   34
    Rosalind_37                     (  96) [r]  229 10.7       1 0.442 0.526   95\end{verbatim}
\end{graycomment}

    \subsection{"Suboptimal Local Alignment": Lalign}
        Το εργαλείο Lalign \cite{Lalign} βρίσκει επαναλαμβανόμενες εσωτερικές ακολουθίες νουκλεοτιδίων ή πρωτεϊνών, στοιχίζοντας ξένες υπακολουθίες, ψάχνοντας ομοιότητες. Για παράδειγμα:
\begin{graycomment} \footnotesize
    \begin{verbatim}
INPUT:
    >Rosalind_48
    GCATA
OUTPUT:
    >Rosalind_48 reverse complement
    TATGC\end{verbatim}
\end{graycomment}


    \subsection{"Base Quality Distribution": Per Base Sequence Quality}
        Το FastQC \cite{FastQC} εμφανίζει διάγραμμα με τη μετρική Base Call Quality. Για παράδειγμα:
    \begin{graycomment} \footnotesize
         \begin{multicols}{2} \centering
             \begin{verbatim}
INPUT:
    @Rosalind_0029
    GCCCCAGGGAACCCTCCGACCGAGGATCGT
    +
    >?F?@6<C<HF?<85486B;85:8488/2/
    @Rosalind_0029
    TGTGATGGCTCTCTGAATGGTTCAGGCAGT
    +
    @J@H@>B9:B;<D==:<;:,<::?463-,,
    @Rosalind_0029
    CACTCTTACTCCCTAGCCGAACTCCTTTTT
    +
    =88;99637@5,4664-65)/?4-2+)$)$
    @Rosalind_0029
    GATTATGATATCAGTTGGCTCCGAGAGCGT
    +
    <@BGE@8C9=B9:B<>>>7?B>7:02+33.
             \end{verbatim}
        \end{multicols}
    \end{graycomment}

    \begin{figure}[H] \noindent \centering
        \includegraphics[scale=0.6]{img/Per Base Sequence Quality}
        \caption{Διάγραμμα μετρικής Base Call Quality.}
    \end{figure}

    \subsection{"Global Multiple Alignment": Clustal}
        Το πρόγραμμα Clustal πραγματοποιεί ολική στοίχιση ακολουθιών.
    \begin{graycomment} \footnotesize
    \begin{verbatim}
OUTPUT:
    Rosalind_7       -----CACGTCTGTTCGCCTAAAACTTTGATTGCCGGCCTACGCTAGTTAGTTA	49
    Rosalind_28      GGGGTCATGGCTGTTTGCCTTAAACCCTTGGCGGCCTAGCCGTAATGTTT----	50
    Rosalind_51      --TCCTATGTTTGTTTGCCTCAAACTCTTGGCGGCCTAGCCGTAAGGTAAG---	49
    Rosalind_18      ---GACATGTTTGTTTGCCTTAAACTCGTGGCGGCCTAGCCGTAAGTTAAG---	48
    Rosalind_23      --ACTCATGTTTGTTTGCCTTAAACTCTTGGCGGCTTAGCCGTAACTTAAG---	49
                           * *  **** **** ****       * *            *\end{verbatim}
    \end{graycomment}

    \subsection{"Finding Genes with ORFs": }
        Το OTF Finder του SMS 2 ξεχωρίζει το κωδικόνιο έναρξης και λήξης και επιστρέφει τη μεγαλύτερη πρωτεϊνική ακολουθία. Για παράδειγμα:
\begin{graycomment} \footnotesize
    \begin{verbatim}
INPUT:
    >sample sequence
    gcggcggcggcggcggcggcggcggcggcggcggcggcggcggcggcggcggcggcggcggcggcggcggcggcggcg...
OUTPUT:
>Translation of ORF number 1 in reading frame 1 on the direct strand.
AAAAAAAAAAAAAAAAAAAAAAAAAAAAAA*\end{verbatim}
\end{graycomment}

    \subsection{"Base Filtration by Quality": }
        Το FASTQ Quality Filter του Galaxy μπορεί να χρησιμοποιηθεί για να καθαρίσουμε τα entries που δεν ικανοποιούν κάποιο threshold.
        Για παράδειγμα για window size 20:
    \begin{graycomment} \footnotesize
    \begin{verbatim}
INPUT:
    @Rosalind_0049
    GCAGAGACCAGTAGATGTGTTTGCGGACGGTCGGGCTCCATGTGACACAG
    +
    FD@@;C<AI?4BA:=>C<G=:AE=><A??>764A8B797@A:58:527+,
OUTPUT:
    @Rosalind_0049
    GCAGAGACCAGTAGATGTGTTTGCGGACGGTCGGGCTCCATGTGACAC
    +
    FD@@;C<AI?4BA:=>C<G=:AE=><A??>764A8B797@A:58:527\end{verbatim}
    \end{graycomment}

    \section{ΒΑΣΕΙΣ ΔΕΔΟΜΕΝΩΝ NCBI \& EBI}
        Η βάση δεδομένων NCBI (National Center for Biotechnology Information) \cite{NCBI} χρησιμοποιεί το COBALT \cite{COBALT} ως εργαλείο πολλαπλής στοίχισης.
        ΤΟ COBALT (Constraint-Based Multiple Alignment Tool) λαμβάνει motif και ομοιότητες από υπάρχουσες βάσεις δεδομένων, τα οποία μετά αξιοποιεί για τη στοίχιση των ακολουθιών.
        Είναι πιο αποτελεσματικό σε συγκεκριμένα είδη πρωτεϊνών, είναι πιο υπολογιστικά απαιτητικό γενικά.

        Αντίθετα, η βάση δεδομένων EBI \cite{EBI} (European Bioinformatics Institute) χρησιμοποιεί το Cluster Omega \cite{ClusterOmega}.
        Το Cluster Omega είναι εξαιρετικά γρήγορο και ευέλικτο καθώς χρησιμοποιεί προοδευτική σύγκριση (progressive alignment) για την κατασκευή πολλαπλών στοιχίσεων.
        Αυτό επιτυγχάνεται με τη δημιουργία ιεραρχικών δομών (guide trees) που αναπαριστούν τις ομοιότητες μέσα στις ακολουθίες, οι οποίες στη συνέχεια στοιχίζονται.
        Μπορεί να στοιχίσει ταυτόχρονα πολλαπλές ακολουθίες, προσφέροντας υψηλή ακρίβεια και κλιμακωτή απόδοση.
\chapter{ΕΡΩΤΗΜΑ 5}
    \vspace{-10pt}
    \begin{center} \noindent
        \includegraphics[scale=0.5]{img/Erwthma5_Tree}
    \end{center}

    Πραγματοποιούμε DFS (Depth-First Search) αναζήτηση από τη ρίζα του δέντρου.
    Κατά τη διάρκεια της αναζήτησης αποθηκεύουμε σε μια μεταβλητή, \texttt{currentPath}, τις συνολικές ετικέτες που διασχίζουμε.
    Σε κάθε κόμβο ελέγχουμε αν το \texttt{currentPath} περιλαμβάνει το πρότυπο \(P\).
    Αν το περιλαμβάνει, κάνουμε μια αναζήτηση στην υπακολουθία μέσω του Knuth-Morris-Pratt αλγορίθμου.

%\begin{lstlisting}[language=Python]
%from Bio.motifs.search import KnuthMorrisPratt
%
%def contains_template(currentPath, P):
%    matches = ""
%
%    # Δημιουργία KMP αντικειμένου με το πρότυπο P
%    kmp = KnuthMorrisPratt(P)
%
%    # Αναζήτηση του προτύπου στο currentPath
%    matches = kmp.search(currentPath)
%
%    if matches == "":
%        return False
%    else:
%        return True
%
%
%def find_template_subpaths(root, P):
%    subpaths = []
%
%    def DFS(node, currentPath):
%        # Append the current node's label to the path
%        currentPath += node.label
%
%        # Check if the current path contains the template P
%        if contains_template(currentPath, P):
%            subpaths.append(currentPath)  # Store the current path
%
%        # Recur for each child
%        for child in node.children:
%            DFS(child, currentPath)
%
%        # Backtrack
%        currentPath = currentPath[:-len(node.label)]
%
%    # Εκκίνηση του DFS
%    DFS(root, "")
%
%    return subpaths
%\end{lstlisting}
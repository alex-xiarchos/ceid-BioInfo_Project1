\chapter{ΕΡΩΤΗΜΑ 5}
    \vspace{-10pt}
    \begin{center} \noindent
        \includegraphics[scale=0.5]{img/Erwthma5_Tree}
    \end{center}

    Έστω πρότυπο \(P\) και κείμενο \(T\).
    Ο μόνος αλγόριθμος που μπορεί να τρέξει realtime, δηλαδή σε \(O(n+m)\) χρόνο, όπου \(n\) ο συνολικός αριθμός χαρακτήρων του \(T\)
        και \(m\) το μήκος του προτύπου \(P\), είναι ο \textbf{Knuth-Morris-Pratt} αλγόριθμος.

    Εφαρμόζουμε τον αλγόριθμο αφού πρώτα διασχίσουμε το δέντρο, όπου για κάθε χαρακτήρα \(\in P \) ακολουθούμε τις ακμές που αντιστοιχούν σε αυτό.
    Μπορεί να χρησιμοποιηθεί κάποιος DFS αλγόριθμος για την εύρεση όλων των εμφανίσεων.